\chapter{Introdução}
\begin{flushright}
	\begin{minipage}{14.0cm}
			\setlength{\parindent}{0cm}
			\textit{Neste capítulo é apresentada a motivação desta monografia, analisando .... Na Seção \ref{sec:regras_recomendacoes} expõe-se brevemente ..... Na Seção \ref{sec:subtitutlo} expõe-se brevemente ..... Na Seção \ref{sec:objetivos} demarcam-se os objetivos deste trabalho. A Seção \ref{sec:motivacao} contém a motivação do trabalho realizado na monografia. E na Seção \ref{sec:organizacao_trabalho} é fornecida uma visão dos capítulos da monografia. \textbf{(item opcional)}}
	\end{minipage}
\end{flushright}

\section{Regras e recomendações}\label{sec:regras_recomendacoes}
Escrever seu texto aqui. De acordo com a \cite{NBR14724:2011} ``o texto é composto de uma parte introdutória, que apresenta os objetivos do trabalho e as razões de sua elaboração; o desenvolvimento, que detalha a pesquisa ou estudo realizado; e uma parte conclusiva''

Para citação direta se deve observar a quantidade de linhas, onde até três linhas ou curta a citação deve ser inserida no parágrafo entre aspas, mas se a citação possui mais de três linhas ou longa esta deve aparecer em parágrafo distinto, com recuo de 4 centímetros da margem esquerda, sem espaçamento, sem aspas e em fonte 10. Outras regras de citações podem ser observadas na \cite{NBR10520:2002}.

A fonte de todos os parágrafos de texto devem ser Times 12, com espaçamento 1,5 e recuo de primeira linha em 1,5.

A nomenclatura dos elementos textuais fica a critério do autor.

Todos títulos de capítulos devem ser negrito, numerado, Times New Roman 20, alinhados à esquerda, espaçamento 1,5, com 36 antes e 18 depois. As seções podem ter mais três níveis também numerados, ou seja, até 1.1.1.1, todas mantém o mesmo padrão de espaçamento (simples, com 36 antes e 18 depois) e alinhamento à esquerda, sendo o a diferença em:
\begin{itemize}
	\setlength\itemsep{-0.17cm}
	\item Subtítulo (Seção \ref{sec:subtitutlo}): normal, Times New Roman 18;
	\item Subassunto (Subseção \ref{subsec:subassunto}): normal, Times New Roman 16; e
	\item Subitem (Subseção \ref{subsubsec:subitem}): normal, Times New Roman 14
\end{itemize}

\section{Subtítulo}\label{sec:subtitutlo}
Descrever aqui o subtitulo.
\subsection{Subassunto}\label{subsec:subassunto}
Descrever aqui o subassunto;
\subsubsection{Subitem}\label{subsubsec:subitem}
Descrever aqui o subitem

\section{Objetivos}\label{sec:objetivos}
Nessa Seção os objetivos serão descritos.
\subsection{Objetivos Gerais}\label{subsec:objetivos_gerais}
Descrever aqui os objetivos gerais.
\subsection{Objetivos Específicos}\label{subsec:objetivos_especificos}
\begin{itemize}
	\setlength\itemsep{-0.17cm}
	\item Objetivo específico 1
	\item Objetivo específico 2
	\item Objetivo específico 3
	\item Objetivo específico 4
\end{itemize}

\section{Motivação}\label{sec:motivacao}
Descreva aqui a motivação do trabalho.

\section{Organização do Trabalho}\label{sec:organizacao_trabalho}
Descreva aqui a organização do trabalho.